



\documentclass[12pt,openany]{scrbook}

% Users of the {thebibliography} environment or BibTeX should use the
% scicite.sty package, downloadable from *Science* at
% www.sciencemag.org/about/authors/prep/TeX_help/ .
% This package should properly format in-text
% reference calls and reference-list numbers.

%\usepackage{scicite}


\usepackage[utf8]{inputenc}


\usepackage{amsmath}
\usepackage{parskip} %sirve para quitar la sangria de los parrafos.
\usepackage{mathrsfs} %sirve para los tipos de letras, por ejemplo en el Lagrangiano L.
\usepackage{graphicx}
\usepackage{wrapfig} %sirve para agregar imagenes en el escrito
\usepackage{caption}
\usepackage{subcaption}
%\usepackage{siunitx}

\usepackage{amssymb}



\newcommand{\dbar}{d\hspace*{-0.08em}\bar{}\hspace*{0.1em}}



%\newtheorem{lemma}{Lema}
\newtheorem{lemma}{Lemma}
\newtheorem{Theorem}{Theorem}
%\newtheorem{Proposition}{Proposici\'on}
\newtheorem{proposition}{Proposition}
\newtheorem{comment}{Comment}

\newcommand{\qedsymbol}{$\blacksquare$}

\newtheorem{definition}{Definition}
%\newtheorem{mydef}{Definici\'on}

\usepackage{ifpdf}
\ifpdf
	\DeclareGraphicsExtensions{pdf,png,jpg}
\else
	\DeclareGraphicsExtensions{.eps}
\fi

\usepackage{anysize}
\marginsize{3cm}{2.8cm}{2.8cm}{2.5cm} % {izquierda}{derecha}{arriba}{abajo}. %para cambiar los margenes



% Use times if you have the font installed; otherwise, comment out the
% following line.

%\usepackage{times}

% The preamble here sets up a lot of new/revised commands and
% environments.  It's annoying, but please do *not* try to strip these
% out into a separate .sty file (which could lead to the loss of some
% information when we convert the file to other formats).  Instead, keep
% them in the preamble of your main LaTeX source file.


% The following parameters seem to provide a reasonable page setup.

%\topmargin 0.0cm
%\oddsidemargin 0.2cm
%\textwidth 16cm 
%\textheight 21cm
%\footskip 0.5cm


%The next command sets up an environment for the abstract to your paper.

\newenvironment{sciabstract}{%
\begin{quote} \bf}
{\end{quote}}


% If your reference list includes text notes as well as references,
% include the following line; otherwise, comment it out.

%\renewcommand\refname{References and Notes}

% The following lines set up an environment for the last note in the
% reference list, which commonly includes acknowledgments of funding,
% help, etc.  It's intended for users of BibTeX or the {thebibliography}
% environment.  Users who are hand-coding their references at the end
% using a list environment such as {enumerate} can simply add another
% item at the end, and it will be numbered automatically.

%\newcounter{lastnote}
%\newenvironment{scilastnote}{%
%\setcounter{lastnote}{\value{enumiv}}%
%\addtocounter{lastnote}{+1}%
%\begin{list}%
%{\arabic{lastnote}.}
%{\setlength{\leftmargin}{.22in}}
%{\setlength{\labelsep}{.5em}}}
%{\end{list}}


% Include your paper's title here

\title{Notas de Python3} 


% Place the author information here.  Please hand-code the contact
% information and notecalls; do *not* use \footnote commands.  Let the
% author contact information appear immediately below the author names
% as shown.  We would also prefer that you don't change the type-size
% settings shown here.

\author
{Jos\'e F\'elix Salazar\footnote{jose.salazar@umich.mx}
\\
\normalsize{%$^{1}$
Instituto de F\'isica y Matem\'aticas, Universidad Michoacana de San Nicol\'as de Hidalgo}\\
\normalsize{Edificio C-3 , Ciudad Universitaria, 58040 Morelia, Michoac\'an, M\'exico.}\\
%\normalsize{$^{2}$Another Unknown Address, Palookaville, ST 99999, USA}\\
\\
%\normalsize{$^\ast$To whom correspondence should be addressed; E-mail:  jsmith@wherever.edu.}
}


 % Include the date command, but leave its argument blank. Sirve para actualizar la fecha.

%%%%%%%%%%%%%%%%% END OF PREAMBLE %%%%%%%%%%%%%%%%


\begin{document} 

\maketitle

\section{Comandos B\'asicos para Nevegar por la Terminal}

\begin{itemize}

\item[•] cd (change directory): Movernos entre distintos directorios\\

\hspace{1cm}	\textbf{cd} directorio

\hspace{1cm}	:$\thicksim$\vspace{0cm}$\$$ \textbf{cd} lpthw/	

para ir un directorio atr\'as hay que usar punto punto .. 

\hspace{1cm}	 cd ..

\item[•] ls (list): Lista todos los archivos del directorio. Algunas variaciones son:

\hspace{1cm} ls -a -- Presenta el contenido incluyendo archivos ocultos (que empiezan con .) 

\hspace{1cm} ls -l -- Muestra el contenido en forma detallada

\hspace{1cm} ls -lh -- Lo mismo que el anterior solo que muestra el tama\~no de cada archivo en
unidades legibles

\hspace{1cm}	:$\thicksim$\vspace{0cm}$\$$ ls

\hspace{1cm}	:$\thicksim$\vspace{0cm}$\$$ ls -a


\item[•] pwd (print working directory): Nos indica el directorio en el que nos encontramos
parados

\hspace{1cm}	$\$$ pwd \\

\item[•] mv (move): Mover archivos entre distintos directorios, al mover un archivo del directorio \textit{dir1} al directorio \textit{dir2} este desaparecer\'a del primer directorio

\hspace{1cm}	$\$$ mv directorio\_origen directorio\_destino

\item[•] cp (copy): copia un archivo de un directorio a otro y a diferencia del mv no borra el
archivo original

\hspace{1cm} $\$$ cp directorio\_origen directorio\_destino 

\item[•] mkdir (make directory): Crea un directorio

\hspace{1cm} $\$$ mkdir nombre\_directorio


\end{itemize}	


\subsection{Abrir Archivos desde Terminal}
Para abrir un archivo desde la terminal, nos
dirigimos a la carpeta donde esta el archivo y despu\'es
usamos el comando \textbf{xdg-open}:

\hspace{1cm} $\$$ xdg-open lp3.pdf

\'o si no estas en la carpeta que contiene al archivo, 
solo debes escribir la ruta que lo contiene, 

\hspace{1cm} $\$$ xdg-open $\thicksim$/home/felix/lpthw/lp3.pdf







\section{Sentencia if}

En la terminal (interprete)

felix@PhD-HP-240-G6-Notebook-PC:$\thicksim$\vspace{0cm}$\$$   python3 \\
$>>>$ a $= 10$ \\
$>>>$ if a $>= 10$: \\
$...$ \qquad \quad print('True') \\
$...$ else: \\
$...$  \qquad \quad print('False') \\
$...$ \\
True


$>>>$ b $= 5$ \\
$>>>$ if b $> 2$: \\
$...$ \qquad \quad print('True') \\
$...$ 
True



\section{M\'etodos}


\subsection{Python 3 - List append() Method}

Descripci\'on:\\
El m\'etodo append() agrega un obj pasado a la lista existente.

Sintaxis:\\
list.append(obj)

Par\'ametros:\\
\textbf{obj} -- Este es el objeto que se agregar\'a a la lista.

Valor devuelto\\
Este m\'etodo no devuelve ning\'un valor pero actualiza la lista existente.

Example:
(En terminal)

felix@PhD-HP-240-G6-Notebook-PC:$\thicksim$\vspace{0cm}$\$$   python3 \\
$>>>$ lista $= [1,2,3,1,1,2,1]$\\
$>>>$ lista.append(4)\\
$>>>$ lista\\
$[1, 2, 3, 1, 1, 2, 1, 4]$


\subsection{Python 3 - List insert() Method}

Descripci\'on:\\
El m\'etodo insert() inserta el objeto obj en la lista en el \'indice de desplazamiento.

Sintaxis:\\
list.insert(index, obj)

Par\'ametros:\\
	\textbf{index} -- Este es el \'indice donde se debe insertar el objeto obj.\\
	\textbf{obj} -- este es el objeto que se insertará en la lista dada.

Valor devuelto:\\
Este método no devuelve ningún valor pero inserta el elemento dado en el \'indice dado.

Ejemplo:
(En terminal)

felix@PhD-HP-240-G6-Notebook-PC:$\thicksim$\vspace{0cm}$\$$   python3 \\
$>>>$ lista = [1,2,3,1,1,2,1]\\
$>>>$ lista.insert(0,0)\\
$>>>$ lista \\
$[0, 1, 2, 3, 1, 1, 2, 1, 4]$ \\
$>>>$ lista.insert(0,-1)\\
$>>>$ lista \\
$[-1, 0, 1, 2, 3, 1, 1, 2, 1, 4]$




\subsection{Python3 - List count() Method}
Descripci\'on:\\
El m\'etodo count() devuelve el n\'umero de apariciones de subcadena en el rango [inicio, fin].

Sintaxis:\\
list.count(obj)

Par\'ametros:\\
\textbf{obj} -- Este es el objeto que ser\'a contado en la
lista.

Ejemplo:
(En terminal)\\

felix@PhD-HP-240-G6-Notebook-PC:$\thicksim$\vspace{0cm}$\$$   python3 \\
$>>>$ lista = [1,2,3,1,1,2,1]\\
$>>>$ lista.count(1)\\
$4$
\\

$>>>$ lista = [1,2,3,1,1,2,1]\\
$>>>$ lista.count(2)\\
$2$


\section{Ambiente Virtual venv}

\subsection{Activar el Ambiente Virtual}

Para activar el ambiente virtual (venv) desde la terminal, 
desde cualquier repositorio en que nos encontremos, debemos
de ejecutar lo siguiente:

\hspace{1cm} $ \$ $ source $\backsim$/home/felix/lpthw/py3/bin/activate

en donde py3 es el nombre de nuestro ambiente virtual. En caso de estar
en el repositorio donde esta nuestro ambiente virtual, solo escribimos

\hspace{1cm} $ \$ $ source py3/bin/activate



\end{document}